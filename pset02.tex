\documentclass[a4paper]{exam}

\usepackage{amsmath,amssymb, amsthm}
\usepackage{geometry}
\usepackage{graphicx}
\usepackage{hyperref}
\usepackage{titling}



% Header and footer.
\pagestyle{headandfoot}
\runningheadrule
\runningfootrule
\runningheader{CS/MATH 113, SPRING 2026}{Pset 02: Predicate Logic}{\theauthor}
\runningfooter{}{Page \thepage\ of \numpages}{}
\firstpageheader{}{}{}

% \printanswers %Uncomment this line

\title{Problem Set 02: Predicate Logic}
\author{Blingblong} % <=== replace with your student ID, e.g. xy012345
\date{CS/MATH 113 Discrete Mathematics\\Habib University\\Spring 2026}

% \qformat{\large\bf \thequestion.}
\boxedpoints

\begin{document}
\maketitle

\begin{questions}
    \question What is the difference between a predicate and a proposition?
    \begin{solution}
        % Enter solution here
    \end{solution}

    \question Consider the following predicates:
    \begin{itemize}
        \item $P(x):$ $x$ an even number.
        \item $Q(x):$ $x$ is a prime number.
        \item $R(x,y):$ $x \geq y$.
        \item $S(x,y):$ $x$ is divisible by $y$.
        \item $T(x):$ $x$ is odd. 
    \end{itemize}
    The domain for quantifiers is the set of all natural numbers.
    Convert the following english statements in predicate logic.
    \begin{parts}
        \part There is a natural number that is even but not divisible by 3.
        \begin{solution}
            % Enter solution here
        \end{solution}

        \part All natural numbers are greater than 0 and if a natural number is divisible by 2 then it is even.
        \begin{solution}
            % Enter solution here
        \end{solution}

        \part There is a unique (only one) natural number that is even and prime.
        \begin{solution}
            % Enter solution here
        \end{solution}

        \part There is no natural number that is both even and odd. 
        \begin{solution}
            % Enter solution here
        \end{solution}

        \part All natural numbers that are either even or odd but not both. If a natural number is even and greater than equal to 3 then if its prime then its not divisible by 2.
        \begin{solution}
            % Enter solution here
        \end{solution}
    \end{parts}

    \question Consider the following predicates:
    \begin{enumerate}
        \item $D(x):$ $x$ is taking the Discrete math course.
        \item $G(x):$ $x$ submits assignments using Gen AI.
        \item $R(x):$ $x$ is a TA, RA or an Instructor of Discrete math.
        \item $Z(x):$ $x$ will get a 0, bcs they did not try.
        \item $H(x):$ $x$ is happy.
        \item $T(x):$ $x$ is having trouble with some problem.
        \item $A(x,y):$ $x$ asks $y$ for help.
        \item $K(x):$ $x$ will come for your knees.
    \end{enumerate}
    The domain for quantifiers consists of everyone in UHU (Univerity of Habib University).
    Convert each of the statements below in predicate logic, and simply where ever you can.
    \begin{parts}
        \part If all the Discrete math students does not submit assignments using Gen AI then all the RAs, TAs and Instructors will be very happy. And if a student is having trouble with some problem and they ask a TA, RA or an Instructor for help then all the RAs, TAs and Instructors will be very happy.
        \begin{solution}
            % Enter solution here
        \end{solution}

        \part If any student is having trouble with some problem, and they didn't ask any TA, RA or an Instructor for help and submits assignments using Gen AI, then that student will will get a 0, bcs they did not try, and Abbas will come for your knees. 
        \begin{solution}
            % Enter solution here
        \end{solution}
        
    \end{parts}

    \question Consider the following predicates:
    \begin{itemize}
        \item $P(x):$ $x$ is a true Sigma.
        \item $Q(x):$ $x$ mews constantly.
        \item $R(x):$ $x$ is a skibidi rizzler.
        \item $S(x,y):$ $x$ will give $y$ knee surgery tomorrow.
        \item $T(x):$ $x$ works at the RO. 
        \item $U(x):$ $x$ is a student in your class. 
        \item $V(x):$ $x$ will get A$^+$ in DM. 
    \end{itemize}
    The domain for quantifiers consists of everyone in UHU (Univerity of Habib University).
    Convert the following statements in English.
    \begin{parts}
        \part $\forall x T(x) \implies (P(x) \land Q(x) \land R(x))$
        \begin{solution}
            % Enter solution here
        \end{solution}

        \part $\forall x  (U(x) \land (P(x)\lor R(x))) \implies V(x)$
        \begin{solution}
            % Enter solution here
        \end{solution}

        \part $\forall x \neg (\exists y \neg V(x) \land S(y,x))$
        \begin{solution}
            % Enter solution here
        \end{solution}

        \part $\exists x \exists y, T(x) \land U(y) \land (R(y) \lor S(x,y))$
        \begin{solution}
            % Enter solution here
        \end{solution}

        \part $\forall x \exists y (V(x) \iff ((T(y) \land U(x) \land S(x,y)) \lor \neg (\neg P(x) \lor \neg Q(x))))$
        \begin{solution}
            % Enter solution here
        \end{solution}
    \end{parts}

    \question Negate the following statements
    \begin{parts}
        \part $\exists y \forall x (P(y) \implies Q(x))$
        \begin{solution}
            % Enter solution here
        \end{solution}

        \part $\forall x \exists y ((P(x) \land \forall z Q(z)) \lor \neg P(y))$
        \begin{solution}
            % Enter solution here
        \end{solution}

        \part $\not\exists x \forall y \exists z (\neg P(x) \land Q(y)) \lor (P(x)\lor \neg R(y))$
        \begin{solution}
            % Enter solution here
        \end{solution}
    \end{parts}
    
\begin{center}
    \includegraphics[scale = 0.95]{grinch.jpeg}
\end{center}
      
\end{questions}
\end{document}

%%% Local Variables:
%%% mode: latex
%%% TeX-master: t
%%% End:
